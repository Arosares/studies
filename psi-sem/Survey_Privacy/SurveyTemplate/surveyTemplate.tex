\documentclass[a4paper]{llncs}
% Please do not change with the font size, line height, and the vertical spacing

\usepackage{amssymb}
\setcounter{tocdepth}{3}
\usepackage{graphicx}
\usepackage{booktabs} % please use booktabs for tables (toprule, midrule, bottomrule)

\usepackage{hyperref}
\usepackage{url}
\usepackage{array}

% using UTF-8 file format allows you to write umlauts into the document
\usepackage[utf8]{inputenc}
\usepackage[T1]{fontenc}

\urlstyle{rm}

\usepackage[protrusion=true,expansion=false]{microtype}



\begin{document}
\mainmatter

% the title of your topic in title case (special capitalization rules apply, see comment below)
\title{What can (normal) users do to you protect your privacy online? Consider private browsing mode, VPNs, Tor, anti-tracking add-ons? What are relevant use cases and what risks are involved in the various approaches?} 

% Capitalization rules taken from http://static.springer.com/sgw/documents/1121537/application/pdf/SPLNPROC+Author+Instructions_Feb2015.pdf:
% Headings should be capitalized (i.e., nouns, verbs, and all other words except articles, prepositions, and conjunctions should be set with an initial capital) [...] Words joined by a hyphen are subject to a special rule. If the first word can stand alone, the second word should be capitalized. Here are some examples of headings: “Criteria to Disprove Context-Freeness of Collage Languages”, “On Correcting the Intrusion of Tracing Non-deterministic Pro- grams by Software”, “A User-Friendly and Extendable Data Distribution System”, “Multi-flip Networks: Parallelizing GenSAT”, “Self-determinations of Man”.

% Insert your name and student number (only in the final seminar paper):
% \author{Mary Miller (1231234) \and Richard Pollock (your student number)}
\author{} % leave the author empty for surveys (double-blind review process) 
\institute{} % the institute should always be left empty

\maketitle



% \begin{abstract}
% This is not necessary for literature surveys
% \end{abstract}

% Sections are not necessary for literature surveys
% \section{Introduction}
% \label{sec:Introduction}

% Please avoid \\, use paragraphs (two newlines instead)
% Please avoid texttt (use \emph instead)
% You can include figures in your survey (mostly not appropriate). Please use floats for that purpose and reference all figures.
% Please do not start sentences with a \cite; use the name of the author or the first author et al. instead.
% Please check your spelling, punctuation, and typography. For instance, use -- for dashes like this: This finding was -- according to the authors -- a significant finding.
% Please use appropriate quotation marks if you cite text verbatim (mostly rephrasing is better). In LaTeX quoatation marks are not printed with " but as follows: ``text''

% Start your survey here
% Usability <-->

To protect ones privacy, users have a multitude of methods available which complement each other to achieve the common goal.

Aggarwal et. al. state that every modern browser ships with a private browsing mode \cite{Aggarwal:2010:APB:1929820.1929828}. In private browsing mode visited pages, cookies, searches and temporary files are not saved to the local disk, while bookmarks and downloads still are. This helps users to protect their privacy from a local attack vector. However, it is important to note that it only protects if the attacker only has access to the pc after the user exits private browsing. Otherwise the local machine could already be compromised by the attacker, f.e. by using a keylogger.

Another feature that modern browsers provide, either direct or indirect through add-ons, is anti-tracking protection. By web tracking it is possible to track a single user over multiple websites, allowing to create profiles or store interests of the user. Very often these trackers are found in advertisements. Bievola et. al. \cite{Bielova:2017:WTT:3133956.3136067} list several extensions that help minimizing the risk of getting tracked, though the first step should be to disable third party cookies, since the third party can identify users by the cookie id. This, however, can lead to inconvenience while browsing. If you don't want to disable all third party cookies at least one of these extensions should be installed: Adblock Plus, uBlock origin, Disconnect, Ghostery or Privacy Badger. These all use different methods to try to find trackers hidden in cookies and block them. Tough even when blocking all third party cookies, users could still get identified and tracked by their ip addresses.

To help preventing that, virtual private networks (VPN) can be used. A VPN 
is able to tunnel your traffic over a server, only letting the target see the VPN servers ip address \cite{Feilner:2006:OBI:1202604}. Configuring your own VPN server can be a tedious task and when choosing a VPN service provider you have no control over the VPN server, i.e. the provider is able to log everything you do on the Internet. On the other hand, the same holds for your Internet service provider (ISP), if you are not using a VPN.

To circumvent this single point of trust the Tor Project started in 2002 \cite{torproject}. Tor is an acronym for "The onion router". The idea is that whenever you target a website, your message will get encrypted and forwarded over multiple relays \cite{Dingledine:2004:TSO:1251375.1251396}. Since each relay only knows its predecessor and its successor, the network itself is not able to identify you. However, the traffic from an exit node to your target server has to be unencrypted in order to be understood by the server, i.e. whenever you login or provide personal information of yourself that you also used when not browsing via the Tor network, it is possible to match your browsing identities. Therefore it is discouraged to use login services, when using the Tor network.


% This template uses bibtex for the bibliography. You can also use biblatex.

% Please include *complete* references (copy&paste the whole bibtex entries, i.e., for the paper as well as the cross-referenced conference as provided by dblp whenever possible)
% These bibtex entries may need some further editing to make them homogenuous)

\bibliography{literature}
\bibliographystyle{splncs03}
\end{document}




