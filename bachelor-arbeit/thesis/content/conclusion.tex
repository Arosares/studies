\section{Conclusion}

To ensure a healthy software ecosystem, different disclosing techniques have emerged.

In this paper we gave an overview of three different vulnerability disclosure
policies: Full Vendor Disclosure, Immediate Public Disclosure and Hybrid
Disclosure.
We agree with Cavusoglu et al. in \cite{Cavusoglu2007}, that the first two favor
either the vulnerability discoverer or the software vendor. 
According to them, the hybrid approach is the best for society of the presented approaches, as the 
vendor is given an incentive to release a patch without exposing users to 
unnecessary risks.
To standardize disclosure CERT/CC documented a in-depth guide for hybrid disclosure and acts as an neutral intermediary for disclosure upon request.
We then compared patching behavior of open source vendors and vendors of proprietary
software. Different studies came to different conclusions of whether the widely
held belief that open source software is more secure and more responsive to
vulnerabilities is true.

Then the incentives for disclosing vulnerabilities and the different markets
that are available to do so were introduced.
We found that based on the incentives of the finder -- such as economic incentives or gaining reputation in the computer security scene --  different markets are attractive.

Finally we selected examples of vulnerabilities and analyzed and categorized
them according the information presented in this paper. However, we found that the hybrid approach is so wide spread, that data on different approaches is hard to obtain.