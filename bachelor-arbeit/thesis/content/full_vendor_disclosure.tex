\subsection{Full Vendor Disclosure}
\label{sec:full-vendor}

One way to disclose software vulnerabilities is the so called full vendor disclosure~\cite{Cavusoglu2007}.
This disclosure policy states that after a vulnerability was discovered it is only disclosed to the vendor of the software and kept secret from the public.
Full vendor disclosure allows the vendor to develop a patch for the vulnerability in their own time or leave the vulnerability unpatched.

Opponents of this approach criticize that the public may be put at risk when provided with substandard data on security \cite{Cavusoglu2007}.
Often users of software can apply a temporary workaround or additional security measures to not be affected by vulnerabilites.
This is not possible when only the vendor is informed about said vulnerability.

Arora et al. find that the release of a patch for a previously unknown vulnerability increases the number of attacks on that vulnerability \cite{Arora2004}.

Proponents of full vendor disclosure however find that publishing a vulnerability immediately to the public leaves the public likewise at risk \cite{Arora2004}.
Due to no patch being available users are often not able to defend themselves and are likely to be victims of attacks. 