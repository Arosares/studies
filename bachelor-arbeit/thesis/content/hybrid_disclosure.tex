\subsection{Hybrid Disclosure}
\label{sec:hybrid}

The third option for disclosing software vulnerabilities is a hybrid approach of 
full vendor disclosure (see \autoref{sec:full-vendor}) and immediate public 
disclosure (see \autoref{sec:immediate-public}) \cite{Cavusoglu2007}.
Hybrid disclosure is also called \textit{responsible disclosure} 
\cite{Cavusoglu2007} or \textit{coordinated vulnerability disclosure} (CVD) 
\cite{CertGuideCVD}.
Discovered vulnerabilities are not disclosed immediately to the public, but the
software vendor is allowed a certain time frame to develop a patch for the 
vulnerability.
If the vulnerability is unpatched after the deadline has passed the finder publicly discloses the vulnerability.

This disclosure policy has the benefit that neither the identifier of the
vulnerability nor the software vendor can be accused of acting irresponsible
\cite{Cavusoglu2007}.
The software vendor also is given an incentive to patch the vulnerability before
the deadline to not jeopardize companies using the vulnerable software.

Often the Computer Emergency Response Team/Coordination Center -- short CERT/CC -- is
included as an intermediate third party \cite{Cavusoglu2007}.
An identifier first releases the vulnerability to CERT/CC, where the 
vulnerability is verified and the affected vendors are notified.
CERT/CC sets a 45-day grace period for the vendor to patch the vulnerability,
after which it is publicly disclosed regardless of the availability of a patch.