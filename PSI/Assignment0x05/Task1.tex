\section{Part 1: nmap}


\subsection{Find the address of that other host with a nmap ping scan.}

To do a ping scan you can use the -sn flag. For the aggressive spee template -T4 does the job. This results in: \verb|nmap -sn  -T4 10.8.200-209.0-255|

Result:\\
Starting Nmap 7.40 ( https://nmap.org ) at 2018-01-04 20:25 CET
Nmap scan report for 10.8.205.198
Host is up (0.00098s latency).
Nmap done: 2560 IP addresses (1 host up) scanned in 133.49 seconds

So the machine was found at IP-address 10.8.205.198

\subsection{Would nmap’s ping scan have found the host if the administrator of the target host had implemented firewall rules that drop ICMP echo packets?}

Probably not.


\subsection{Perform a nmap TCP connect port scan.}

nmap -sT -T4 10.8.200-209.0-255

\begin{lstlisting}
	Starting Nmap 7.40 ( https://nmap.org ) at 2018-01-04 20:37 CET
	Nmap scan report for 10.8.205.198
	Nmap scan report for 10.8.205.198
	Host is up (0.0027s latency).
	Not shown: 860 closed ports
	PORT      STATE SERVICE
	1/tcp     open  tcpmux
	4/tcp     open  unknown
	6/tcp     open  unknown
	7/tcp     open  echo
	9/tcp     open  discard
	13/tcp    open  daytime
	17/tcp    open  qotd
	19/tcp    open  chargen
	20/tcp    open  ftp-data
	21/tcp    open  ftp
	22/tcp    open  ssh
	23/tcp    open  telnet
	37/tcp    open  time
	42/tcp    open  nameserver
	43/tcp    open  whois
	49/tcp    open  tacacs
	53/tcp    open  domain
	70/tcp    open  gopher
	79/tcp    open  finger
	80/tcp    open  http
	88/tcp    open  kerberos-sec
	106/tcp   open  pop3pw
	110/tcp   open  pop3
	111/tcp   open  rpcbind
	113/tcp   open  ident
	119/tcp   open  nntp
	135/tcp   open  msrpc
	139/tcp   open  netbios-ssn
	143/tcp   open  imap
	161/tcp   open  snmp
	163/tcp   open  cmip-man
	179/tcp   open  bgp
	199/tcp   open  smux
	389/tcp   open  ldap
	406/tcp   open  imsp
	427/tcp   open  svrloc
	443/tcp   open  https
	444/tcp   open  snpp
	445/tcp   open  microsoft-ds
	464/tcp   open  kpasswd5
	465/tcp   open  smtps
	500/tcp   open  isakmp
	512/tcp   open  exec
	513/tcp   open  login
	514/tcp   open  shell
	515/tcp   open  printer
	543/tcp   open  klogin
	544/tcp   open  kshell
	548/tcp   open  afp
	554/tcp   open  rtsp
	563/tcp   open  snews
	636/tcp   open  ldapssl
	749/tcp   open  kerberos-adm
	765/tcp   open  webster
	777/tcp   open  multiling-http
	783/tcp   open  spamassassin
	808/tcp   open  ccproxy-http
	873/tcp   open  rsync
	901/tcp   open  samba-swat
	990/tcp   open  ftps
	992/tcp   open  telnets
	993/tcp   open  imaps
	995/tcp   open  pop3s
	1001/tcp  open  webpush
	1080/tcp  open  socks
	1093/tcp  open  proofd
	1094/tcp  open  rootd
	1099/tcp  open  rmiregistry
	1236/tcp  open  bvcontrol
	1300/tcp  open  h323hostcallsc
	1352/tcp  open  lotusnotes
	1433/tcp  open  ms-sql-s
	1434/tcp  open  ms-sql-m
	1524/tcp  open  ingreslock
	1812/tcp  open  radius
	1863/tcp  open  msnp
	2000/tcp  open  cisco-sccp
	2003/tcp  open  finger
	2010/tcp  open  search
	2049/tcp  open  nfs
	2103/tcp  open  zephyr-clt
	2105/tcp  open  eklogin
	2111/tcp  open  kx
	2119/tcp  open  gsigatekeeper
	2121/tcp  open  ccproxy-ftp
	2135/tcp  open  gris
	2401/tcp  open  cvspserver
	2601/tcp  open  zebra
	2602/tcp  open  ripd
	2604/tcp  open  ospfd
	2605/tcp  open  bgpd
	2607/tcp  open  connection
	2608/tcp  open  wag-service
	2811/tcp  open  gsiftp
	3260/tcp  open  iscsi
	3306/tcp  open  mysql
	3493/tcp  open  nut
	3689/tcp  open  rendezvous
	3690/tcp  open  svn
	4224/tcp  open  xtell
	4899/tcp  open  radmin
	5002/tcp  open  rfe
	5050/tcp  open  mmcc
	5051/tcp  open  ida-agent
	5060/tcp  open  sip
	5061/tcp  open  sip-tls
	5190/tcp  open  aol
	5222/tcp  open  xmpp-client
	5269/tcp  open  xmpp-server
	5555/tcp  open  freeciv
	5666/tcp  open  nrpe
	6000/tcp  open  X11
	6001/tcp  open  X11:1
	6002/tcp  open  X11:2
	6003/tcp  open  X11:3
	6004/tcp  open  X11:4
	6005/tcp  open  X11:5
	6006/tcp  open  X11:6
	6007/tcp  open  X11:7
	6346/tcp  open  gnutella
	6566/tcp  open  sane-port
	6667/tcp  open  irc
	7000/tcp  open  afs3-fileserver
	7001/tcp  open  afs3-callback
	7002/tcp  open  afs3-prserver
	7004/tcp  open  afs3-kaserver
	7007/tcp  open  afs3-bos
	7100/tcp  open  font-service
	8021/tcp  open  ftp-proxy
	8081/tcp  open  blackice-icecap
	8088/tcp  open  radan-http
	9101/tcp  open  jetdirect
	9102/tcp  open  jetdirect
	9103/tcp  open  jetdirect
	9418/tcp  open  git
	10000/tcp open  snet-sensor-mgmt
	10082/tcp open  amandaidx
	13722/tcp open  netbackup
	13782/tcp open  netbackup
	13783/tcp open  netbackup
	
\end{lstlisting}

\subsection{Explain how the disadvantages of the basic TCP connect scan can be overcome by other scan types.}

You could do an Idle Scan using the -sl flag. 
1) A SYN scan will send a SYN - packet to the port and the target will respond with SYN/ACK messages if the port is open and with RST if the port is closed.
2) A machine that receives SYN/ACK packet will repond with a RST.
3) Every IP packet on the Internet has a fragment identification number (IP ID). Since many operating systems simply increment this number for each packet they send, probing for the IPID can tell an attacker how many packets have been sent since the last probe.

By combining these traits, it is possible to scan a target network while forging your identity so that it looks like an innocent zombie machine did the scanning

Idle scan consists of three steps:
\begin{itemize}
	\item Probe zombie's IP ID and record it.
	\item Forge SYN packet from zombie and send it to desired port. Depending on the target's reaction the IP ID may be incremented.
	\item Repeat step 1 again and compare the recordings.
\end{itemize}

The zombie's IP ID should now have increased by either one or two. One means the zombie hasn't sent out any packets, except for its reply to the attacker's probe. This means the port is not open. Two indicates the zombie sent out a packet between the two probes. This usually means the port is open. Note that via this scan method closed and filtered ports can not be distinguished.

See: \url{https://nmap.org/book/idlescan.html}

\subsection{Perform a version detection scan to filter out the dummy ports. On which ports are real services running?}

\begin{lstlisting}
	nmap -sV -F -T4 10.8.200-209.0-255
	Nmap scan report for 10.8.205.198
	Host is up (0.00068s latency).
	Not shown: 48 closed ports
	PORT      STATE SERVICE    VERSION
	7/tcp     open  tcpwrapped
	9/tcp     open  tcpwrapped
	13/tcp    open  tcpwrapped
	21/tcp    open  tcpwrapped
	22/tcp    open  ssh        OpenSSH 7.4p1 Debian 10+deb9u2 (protocol 2.0)
	23/tcp    open  tcpwrapped
	37/tcp    open  tcpwrapped
	53/tcp    open  tcpwrapped
	79/tcp    open  tcpwrapped
	80/tcp    open  http       nginx 1.10.3
	88/tcp    open  tcpwrapped
	106/tcp   open  tcpwrapped
	110/tcp   open  tcpwrapped
	111/tcp   open  tcpwrapped
	113/tcp   open  tcpwrapped
	119/tcp   open  tcpwrapped
	135/tcp   open  tcpwrapped
	139/tcp   open  tcpwrapped
	143/tcp   open  tcpwrapped
	179/tcp   open  tcpwrapped
	199/tcp   open  tcpwrapped
	389/tcp   open  tcpwrapped
	427/tcp   open  tcpwrapped
	443/tcp   open  ssl/http   nginx 1.10.3
	444/tcp   open  tcpwrapped
	445/tcp   open  tcpwrapped
	465/tcp   open  tcpwrapped
	513/tcp   open  tcpwrapped
	514/tcp   open  tcpwrapped
	515/tcp   open  tcpwrapped
	543/tcp   open  tcpwrapped
	544/tcp   open  tcpwrapped
	548/tcp   open  tcpwrapped
	554/tcp   open  tcpwrapped
	873/tcp   open  tcpwrapped
	990/tcp   open  tcpwrapped
	993/tcp   open  tcpwrapped
	995/tcp   open  tcpwrapped
	1433/tcp  open  tcpwrapped
	2000/tcp  open  tcpwrapped
	2049/tcp  open  tcpwrapped
	2121/tcp  open  tcpwrapped
	3306/tcp  open  tcpwrapped
	4899/tcp  open  tcpwrapped
	5051/tcp  open  tcpwrapped
	5060/tcp  open  tcpwrapped
	5190/tcp  open  tcpwrapped
	5666/tcp  open  tcpwrapped
	6000/tcp  open  tcpwrapped
	6001/tcp  open  tcpwrapped
	8081/tcp  open  tcpwrapped
	10000/tcp open  tcpwrapped
	Service Info: OS: Linux; CPE: cpe:/o:linux:linux_kernel
	
\end{lstlisting}

When receiving 'tcpwrapped' it indicates a valid TCP-handshake was performed but then closes the connection. Due to the high amount of tcpwrapped services we can assume that this is a defence mechanism to hide real services. In this case "tcpwrapped" indicates that the service is a dummy.

Real services are running on ports 22, 80 and 443.

\subsection{Scan the port range 10000 to 65535 and determine on which port the web application is running.}

\begin{lstlisting}
	nmap -sT -T4 -p 10000-65535 10.8.200-209.0-255
	
	Starting Nmap 7.40 ( https://nmap.org ) at 2018-01-04 21:04 CET
	Stats: 0:02:56 elapsed; 0 hosts completed (0 up), 2560 undergoing Ping Scan
	Nmap scan report for 10.8.205.198
	Host is up (0.0017s latency).
	Not shown: 55501 closed ports
	PORT      STATE SERVICE
	10000/tcp open  snet-sensor-mgmt
	10050/tcp open  zabbix-agent
	10051/tcp open  zabbix-trapper
	10080/tcp open  amanda
	10081/tcp open  famdc
	10082/tcp open  amandaidx
	10083/tcp open  amidxtape
	10809/tcp open  nbd
	11112/tcp open  dicom
	11201/tcp open  smsqp
	11371/tcp open  pksd
	13720/tcp open  netbackup
	13721/tcp open  netbackup
	13722/tcp open  netbackup
	13724/tcp open  vnetd
	13782/tcp open  netbackup
	13783/tcp open  netbackup
	15345/tcp open  xpilot
	17001/tcp open  unknown
	17002/tcp open  unknown
	17003/tcp open  unknown
	17004/tcp open  unknown
	17500/tcp open  db-lsp
	20011/tcp open  unknown
	20012/tcp open  ss-idi-disc
	22125/tcp open  dcap
	22128/tcp open  gsidcap
	22273/tcp open  wnn6
	24554/tcp open  binkp
	27374/tcp open  subseven
	30865/tcp open  unknown
	55329/tcp open  unknown
	57000/tcp open  unknown
	60177/tcp open  unknown
	60179/tcp open  unknown
	
	Nmap done: 2560 IP addresses (1 host up) scanned in 275.04 seconds
	
\end{lstlisting}

The desired machine it at this address: http://10.8.205.198:55329/