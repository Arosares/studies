\section{Part 2: Brute forcing a login of a web application}

\subsection{How is form data sent from the browser to the server?}

Form contents are expressed as a property list of attribute names and values. This can for example be achieved as a suffix on the URL given by the 'ACTION' attribute. The list will be encoded as sequence of name=value elements separated by the '\&' character. Example: \verb|URL?org=Acme%20Foods&commerce&users=42|



\subsection{Analyze the login form with the developer tools of your browser.}

First we created a ssh-tunnel via \verb|ssh -L 15900:10.8.205.198:55329 -l <user> 88.99.184.129| and connected to it from our browser via \verb|127.0.0.1:15900|

When just reloading the page an error "Security token does not match" appears.



\subsection{Your final objective is to find a working pair of username and password with which you can log into the web server.}

Due to time limitations we were not able to find the correct username and password to log in.

To bruteforce the password we would write a script with nested for loops going from 1 to 31 and from 1 to 12 respectively and appending m and f at each string.
Then this will be inserted as password.