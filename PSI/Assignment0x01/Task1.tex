\section{Task 1: Account Recovery}

\subsection{Most important recommendations for the account recovery process}

\paragraph{General}
\begin{itemize}
	\item Don't store passwords in plain text, but checksums.
	\item Balance ease of use with challenge.
	\item Regularly revalidate information like current email address, phone numbers etc.
\end{itemize}

\paragraph{Email Authentication}
\begin{itemize}
	\item Focus on recovering the account - not all features
	\item Don't send their old password, but a new, randomly generated password. Or a token.
	\item Notify real account holder over all channels available.
\end{itemize}

\paragraph{Knowledge-Based}
\begin{itemize}
	\item Don't allow easily guessed information.
	\item Plan for cultural differences.
	\item Changes should require at least as much authority as recovery.
\end{itemize}

\paragraph{Social}
\begin{itemize}
	\item Don't spam or otherwise alienate the users social contacts required for information.
\end{itemize}

\paragraph{Multi-Channel Authentication}
\begin{itemize}
	\item Is only useful if independent, so sending a SMS to a phone that is used to recover the account is not helpful.
\end{itemize}

\subsection{Example: dotasource.de}

As test object I used the web site \url{www.dotasource.de}.

This is what the Lost Password page looks like:

\begin{figure}[H]
	\centering
	\includegraphics[width=0.8\textwidth]{Assignment0x01/image/recover_1.png}
	\caption{Lost Password} \label{img:recover_1}
\end{figure}

When entered a non-existent username or E-mail address the page responds that the user does not exist or no account with that e-mail is registered.

\begin{figure}[H]
	\centering
	\includegraphics[width=0.8\textwidth]{Assignment0x01/image/recover_2.png}
	\caption{User does not exist} \label{img:recover_2}
\end{figure}

\begin{figure}[H]
	\centering
	\includegraphics[width=0.8\textwidth]{Assignment0x01/image/recover_3.png}
	\caption{E-mail not used by an account} \label{img:recover_3}
\end{figure}

When entering correct information an email is sent to the account's address, containing a link for recovery.

\begin{figure}[H]
	\centering
	\includegraphics[width=0.8\textwidth]{Assignment0x01/image/email_with_request_link_blacked}
	\caption{Reset Link} \label{img:link_blacked}
\end{figure}

When clicked on the link another email with a new, random generated password in plain text is sent.
Sessions that are still logged in don't lock you out. But for a new login the password is in immediate effect.
To change the password again, you have to login using the given password and visit the account management site.

What is definitely missing is that it is never asked if the given email address is still valid. Also it should be changed that the system responds that the user/e-mail does not exist. Therefore a message like "If the user exists, an email was sent to his address" would be more suitable. Though given this is a public forum, it still would be possible to create an account and look at all registered user names.
