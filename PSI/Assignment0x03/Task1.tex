\section{A Ciphertext Only Attack on the Vigenère Cipher}

First we used the Kaisiki-Test do determine the Key length:\\
\newline
\begin{tabular}{l | c | r}
	letters & distance & prime factors\\
	\hline
	KC:&  41 & 41\\
	CM:&  17 & 17\\
	SY:& 66 & 2  *  3  * 11\\
	XOC:& 140 & 2  *  2  *  5  *  7\\
	OC:& 20 & 2  *  2  *  5\\
	GK:& 12 & 2  *  2  *  3\\
	JO:& 12 & 2  *  2  *  3\\
	JO:& 42 &	2 * 3 * 7\\
	JO:& 16 & 2 * 2 * 2 * 2\\
	JO:& 74 & 2 * 37\\
	JO:& 136 & 2*2*2*17\\	
	LZKMP & 520 & 2*2*2*5*13\\
	LZKMP & 178 & 2 * 89\\
	LZKMP & 366 & 2*3*61\\
	LKZMP & 1448 & 2*2*2*181\\  
\end{tabular}\\
\newline
Because of the overwhelming amount of twos we decided to try 4 as key length.
Then we did a frequency analysis on the partitions created by \url{https://cryptotools.psi.h4q.it/vigenere.html}.
The most frequent letters for each partition were:\\
\newline
1) P, E, L, Z, S\\
2) S, H, B, C, O\\
3) G, V, Q, C, J, K\\
4) O, D, K, Y, C\\


After trying a little bit around, we discovered that the word L O C K is readable when taking one character from every partition. We tried it out and had success. The key for the text is LOCK and Alice finds a golden key!
