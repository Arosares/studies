\section{Introduction}

Cloud computing is a old paradigm and ideas that went in the same direction exist since the 1990s. Since then many new technologys emerged that first made the cloud possible at all and then forced it to adapt to new standards. From being super centralized we went all the way to a very distributed infrastructure, but now we begin to face new challenges. Challenges about the huge masses of data we generate and that need to be processed, about our privacy and how we can decide what data we want to publish. Challenges of network congestion because you can't send unlimited amounts of data through the pipes. To face and to try to solve these and many more challenges a new mutation of cloud computing is emerging. Fog Computing takes a step back from the super distributed approach and lets data get processed and filtered near the end user so that only relevant data will be pushed upstream. The technical progress in hardware allows even the most basic devices to process data with less latency than sending raw data to the cloud and waiting for it to come back. In this paper I want to give a overview over the chances of Fog Computing, how it integrates within the cloud and how it helps to solve rising challenges in the field of the Internet of Things.