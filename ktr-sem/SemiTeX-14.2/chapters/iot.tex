\section{Fog and IoT}

\subsection{Rising Challenges of IoT}
The Internet of Things is a very promising but also very criticised field.
Approximately 17 - 30 billion IoT devices are assumed to be online by 2020\footnote{http://spectrum.ieee.org/tech-talk/telecom/internet/popular-internet-of-things-forecast-of-50-billion-devices-by-2020-is-outdated}, in cases where it makes sense and where it doesn't. While for investors this can be a huge opportunity, these devices open a very big attack surface and implementing the fog architecture can make things a lot easier. Chiang is naming 5 different main challenges \cite[p. 855 ff.]{DBLP:journals/iotj/ChiangZ16}:

\begin{itemize}
	\item [A] Stringent Latency Requirements: Control systems require latencies between sensor and control node to be within few milliseconds. Some even within tens of a millisecond.
	\item [B] Network Bandwidth Constraints: Large amount of data generated are generated by sensors(e.g.: up to one gigabyte per second for an autonomous car) requiring high bandwidth if sent to the cloud.
	\item [C] Resource-Constrained Devices: IoT devices and micro-controllers often have only the resources they need to work. Everyone of them communicating directly with the cloud will cause a lot of overhead.
	\item [D] Cyber Physical Systems: Cyber-physical systems are often required to be online all the time, especially in safety-critical areas.
	\item [E] Uninterrupted Services With Intermittent Connectivity to the Cloud: Functionality of a system has to be working even without network connectivity.
    \item [F] New Security Challenges:
	\begin{itemize}
		\item[1)] Keeping Security Credentials and Software up to Date on Large Number of Devices.
		\item[2)] Protecting Resource-Constrained Devices.
		\item[3)] Assessing the Security Status of Large Distributed Systems in Trustworthy Manner.
		\item[4)] Responding to Security Compromises Without Causing Intolerable Disruptions.
	\end{itemize}
\end{itemize}

\subsection{How Fog helps to deal with these challenges}\label{sec:fog_challenges}

The main advantages of the emergent fog architecture are often illustrated as CEAL \cite[p. 858]{DBLP:journals/iotj/ChiangZ16} and \cite[p. 7]{OpenFog}:

\begin{itemize}
	\item Cognition: A fog node is aware of its environment to react fast to events and if needed redistribute the workload in an optimal manner.
	\item Efficiency: Dynamically using and redistributing unused resources from end user devices
	\item Agility: Rapid innovation and affordable scaling.
	\item Latency: Real-time processing and cyber-physical system control.
\end{itemize}

\subsubsection{Dynamic discovery}
The Internet of Things is a very heterogeneous field. There is a large variety in sensors, that output  different kinds of data over different kinds of communication protocols. That's why a adequate architecture needs to be capable of dynamic discovery and connection of edge devices while providing a high security standard.\\
The discovering of edge nodes can function in two different ways (see )\autoref{img:2_way_discover}):
\begin{itemize}
	\item By letting the edge nodes search continuously for fog gateways which wait for a connection request
	\item By letting the fog gateway search for edge nodes in range and leave the edge devices discoverable
\end{itemize}

\image{8cm}{2_way_discovery.PNG}{Two ways of discovering edge nodes\cite[p. 16]{Perera:2017:FCS:3101309.3057266}}{img:2_way_discover}

Both methods usually use low power communication protocols which will be described in detail in section \ref{sec:comm}

% In order to achieve this two things need to happen. First, the edge devices need to be connected to a fog node. Second, the fog node needs to be connected to a IoT cloud platform.\cite[p. 15]{Perera:2017:FCS:3101309.3057266}

\subsubsection{Latency}

Due to evaluating and gathering data closer to the end user, systems can react faster on changing circumstances. This way information does not need to travel to a distant server and back, saving bandwidth and time.

\subsubsection{Network Bandwidth}

Fog controller nodes are capable of distributing the workload along the edge node to cloud continuum in a optimal manner, resulting in less data being sent over the network.


\subsubsection{Uninterrupted Service}

A Fog node is independent of the Internet. It is still able to get data or distribute tasks to its connected edge devices while being disconnected from the rest of the world. This allows local services to stay up even if the node is disconnected.

\subsubsection{Security}

To ensure secure connections from an edge node to the cloud all "things" must "employ a hardware-based immutable root of trust."\cite[p.10]{OpenFog} i.e. to ensure security the hardware itself needs to be protected from unauthorized manipulation. Then on top of that software agents running throughout the infrastructure must be able to ensure the integrity of the root of trust. Due to the nature of edge nodes in being close to the end user, fog nodes often act as first node of access control and encryption enabling the user to decide what privacy-sensitive data should be aggregated before it leaves the node.\cite[p.10]{OpenFog}