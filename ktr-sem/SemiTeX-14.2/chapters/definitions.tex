\section{Definitions}
\subsection{Cloud Computing}
"The cloud is just someone else's computer" is a wide-spread joke. And while it can be correct, it is not complete. According to the NIST, cloud computing is a "model for enabling ubiquitous, convenient, on-demand network access to a shared
pool of configurable computing resources (e.g., networks, servers, storage, applications, and services) that
can be rapidly provisioned and released with minimal management effort or service provider interaction."\cite[S. 2]{CloudNIST11}
Additionally the model is defined by five essential characteristics, three service models and four deployment models:

Essential Characteristics:
\begin{itemize}
	\item On-demand self-service: A consumer can provision computing capabilities such as server time and network storage as needed automatically without requiring human interaction with each service provider.
	\item Broad network access: Capabilities are available over the network accessed through standard mechanism that promote use by heterogeneous thin or dick client platforms (e.g., mobile phones, tablets, laptops, and workstations).
	\item Resource pooling: The provider's computing resources are pooled to serve multiple consumers using a multi-tenant model, with different physical and virtual resources dynamically assigned and reassigned  according to consumer demand. There is a sense of location independence in that the customer generally has no control or knowledge over the exact location of the provided resources but may be able to specify location at a higher level of abstraction (e.g., country, state, or datacenter). Examples of resources include storage,
	processing, memory, and network bandwidth.
	\item Rapid elasticity: Capabilities can be elastically provisioned and released, in some cases	automatically, to scale rapidly outward and inward commensurate with demand. To the	consumer, the capabilities available for provisioning often appear to be unlimited and can be appropriated in any quantity at any time.
	\item Measured service: Cloud systems automatically control and optimize resource use by leveraging	a metering capability1 at some level of abstraction appropriate to the type of service (e.g., storage, processing, bandwidth, and active user accounts). Resource usage can be monitored, controlled, and reported, providing transparency for both the provider and consumer of the utilized service.
\end{itemize}

Service Models:
\begin{itemize}
	\item Software as a Service (SaaS): The capability provided to the consumer is to use the provider’s applications running on a cloud infrastructure. The applications are accessible from various client devices through either a thin client interface, such as a web browser (e.g.,	web-based email), or a program interface. The consumer does not manage or control the	underlying cloud infrastructure including network, servers, operating systems, storage, or even individual application capabilities, with the possible exception of limited userspecific application configuration settings.
	\item Platform as a Service (PaaS): The capability provided to the consumer is to deploy onto the cloud	infrastructure consumer-created or acquired applications created using programming languages, libraries, services, and tools supported by the provider. The consumer does not manage or control the underlying cloud infrastructure including network, servers, operating systems, or storage, but has control over the deployed applications and possibly configuration settings for the application-hosting environment.
	\item Infrastructure as a Service (IaaS): The capability provided to the consumer is to provision	processing, storage, networks, and other fundamental computing resources where the consumer is able to deploy and run arbitrary software, which can include operating systems and applications. The consumer does not manage or control the underlying cloud infrastructure but has control over operating systems, storage, and deployed applications;	and possibly limited control of select networking components (e.g., host firewalls).
\end{itemize}

Deployment Models:

\begin{itemize}
	\item Private cloud: The cloud infrastructure is provisioned for exclusive use by a single organization comprising multiple consumers (e.g., business units). It may be owned, managed, and operated by the organization, a third party, or some combination of them, and it may exist on or off premises.
	\item Community cloud: The cloud infrastructure is provisioned for exclusive use by a specific	community of consumers from organizations that have shared concerns (e.g., mission, security requirements, policy, and compliance considerations). It may be owned, managed, and operated by one or more of the organizations in the community, a third	party, or some combination of them, and it may exist on or off premises.
	\item Public cloud: The cloud infrastructure is provisioned for open use by the general public. It may be owned, managed, and operated by a business, academic, or government organization, or some combination of them. It exists on the premises of the cloud provider.
	\item Hybrid cloud: The cloud infrastructure is a composition of two or more distinct cloud infrastructures (private, community, or public) that remain unique entities, but are bound together by standardized or proprietary technology that enables data and application portability (e.g., cloud bursting for load balancing between clouds).
\end{itemize}
(\cite[see s. 2f]{CloudNIST11})

\newpage
\subsection{Fog Computing}\label{sec:def_fog}

"Fog is an emergent architecture  for computing, storage, control and networking that distributed these services closer to the end-users along the cloud-to-things continuum". \cite[S. 854]{DBLP:journals/iotj/ChiangZ16}. However, Fog Computing is not a strict architecture that has a clear definition, "instead it represents a notion that supports to push data analyrics towards leaves(i.e. edge nodes)"\cite[p. 3]{Perera:2017:FCS:3101309.3057266}. Chiang and Perera et al. both stress that the important part of the definition is that tasks are not supposed to be completed by the edge nodes only, but on every node on the path to the cloud server as needed by the use case.



\image{8cm}{fog-and-cloud.png}{Fog-interfaces}{img:fog}\cite[p. 862]{DBLP:journals/iotj/ChiangZ16}

As seen in \autoref{img:fog} Fog is not designed to replace the cloud, but to play along with it in order to solve rising challenges, especially in the field of IoT. The end point of such a network is called fog node and each one has at least one computing server to which all of its edge devices connect to. The Fog node controller than merges, stores, computes or redistributes data gathered or created by the edge devices. It is also capable of assigning tasks to the computing power of the edge devices. The node itself can either be controlled by another controller node that manages all edge nodes or commits to send selected results directly to a cloud server.

\subsection{Glossary}

A Fog Node describes the combination of multiple edge nodes/IoT devices/sensors and a fog node controller which acts as gateway. However Fog Nodes can also inherit multiple other fog nodes as seen in \autoref{img:f2c}. Since a sketch of a fog architecture often looks like a tree, fog nodes at the edge of a network are usually referred as leave nodes.

\image{9cm}{f2c.png}{Fog2Cloud Architecture\cite{DBLP:journals/corr/Marin-TorderaMA16}}{img:f2c}

